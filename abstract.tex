\begin{abstract}
\thispagestyle{plain} 
\setcounter{page}{2}
Thread-Level Speculation (TLS) is one of the parallel frameworks. TLS can avoid the analysis problem of compiler-directed code parallelization and this is helpful for programmers to generate parallel programs. However, the performance is the most important issue for parallel programs. Therefore, we analyse the performance of hardware Thread-Level Speculation (TLS) in the IBM Blue Gene/Q computer.

This paper presents a performance model for hardware Thread-Level Speculation (TLS) in the IBM Blue Gene/Q computer. The model shows good performance prediction, as verified by the experiments. The model helps to understand potential gains from using special purpose TLS hardware to accelerate the performance of codes that, in a strict sense, require serial processing to avoid memory conflicts. Based on analysis and measurements of the TLS behavior and its overhead, a strategy is proposed to help utilize this hardware feature. Furthermore, we compare the performance of hardware Thread-Level Speculation and OpenMP. Based on the performance analysis, we give a direction for deciding between this two parallel frameworks. And the results can not only help users to utilize the TLS but also suggest potential improvement for the future TLS architectural designs.
\end{abstract}

% IEEEtran.cls defaults to using nonbold math in the Abstract.
% This preserves the distinction between vectors and scalars. However,
% if the conference you are submitting to favors bold math in the abstract,
% then you can use LaTeX's standard command \boldmath at the very start
% of the abstract to achieve this. Many IEEE journals/conferences frown on
% math in the abstract anyway.

% no keywords
