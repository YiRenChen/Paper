\chapter{Background}
\label{chap:background}

Now we do face a revolution in data processing as internet information size that we can collect grows quickly, currently there are many solutions developed to analyze and mine valuable treasure in diverse and useless garbage data.
Hadoop \cite{5496972} and MapReduce \cite{Dean:2008:MSD:1327452.1327492} are famous and effective solutions, they give programmer user-friendly and compact batch processing module. 

Mapreduce mainly involves batch processing technique on shared and parallel cluster meaning it works statically and event triggered, But MapReduce cannot provide low latency response when it process continous and real-time unbounded data stream \cite{Babcock:2002:MID:543613.543615}.
To overcome this disadvantage, several stream processing system have been proposed, e.g. Storm \cite{Aniello:2013:AOS:2488222.2488267}
 and S4 \cite{5693297}.
The abvious difference between MapReduce and Storm is that a MapReduce work will finally finish, while Storm work will keep on working until be stoped by user.

\section*{Storm}

Storm attributes distributed realtime stream processing framework with scalable, reliable and fault tolerant computation, it can act excellently on many use cases like: realtime analytics, online machine learning, ETL, and more.
For these surprising features, Many companies such as Twitter choose Storm as data processing System.

\subsection*{Components of Storm}

A Storm cluster is superficaly similar to a Hadoop cluster, there are two kinds of nodes on a Storm clster: the master node and worker nodes.
Storm executes a daemon called {\em Nimbus} on master node as Hadoop runs {\em JobTracker} on it, Nimbus mainly manages task assignment, work code dirtribution and failure recovery on cluster.
There will be a daemon called {\em Supervisor} on each worker node, Supervisor's mission is listening job assignment from Nimbus and manages worker processes on it for assigned works.
To achieve coordination and fault tolerancy, Storm use ZooKeeper \cite{Hunt:2010:ZWC:1855840.1855851} as communication interface between Nimbus and Supervisors, which is illustrated in Fig ~\ref{fig:zookeeper cluster}. 
Additionally, Nimbus and Supervisor daemon are fail-fast and stateless; all Strom daemon state is kept in Zookeeper or local disk.
It means if Nimbus or Supervisor are broken, they can be start up like nothing happened.

\begin{figure}[h]
\graphicspath{{figure/}}
\includegraphics[scale=0.65]{storm_physical_cluster.eps}
\caption{the physical views of Storm}
\label{fig:zookeeper cluster}
\end{figure}

\subsection*{Storm Topology}

Although MapReduce and Storm are similiar physically, there are still some differences between them in logical view.
For example, we use {\em MapReduce job} to describe MapReduce work, but in Storm it use another name {\em topology} as alias for Storm work.
A standard Storm topology is described as a directed acyclic graph which usually includes two kinds of components: {\em spout} and {\em bolt}.
Spout is a source of data stream in topology, spout can generate data itself or receive it from  
